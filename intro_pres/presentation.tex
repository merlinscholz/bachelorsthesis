\documentclass[9pt]{beamer}
\usepackage[ngerman]{babel}
\usepackage{bibgerm}

\usetheme{TUDOplain}
\setbeamertemplate{navigation symbols}{}
\setbeamertemplate{footline}{\small \vspace{-1ex} \vbox{ \insertframenumber /\inserttotalframenumber}}

\author[Merlin Scholz]{Merlin Scholz\\\href{mailto:merlin.scholz@tu-dortmund.de}{merlin.scholz@tu-dortmund.de}}
\title[Analyse der Marsoberfläche durch Unsupervised Learning]{Kategorisieren der Marsoberfläche durch Unsupervised Learning by Backpropagation}
\date[20.11.2019]{20. November 2019}
\institute[TU Dortmund]{Mustererkennung,\\ Informatik XII, Technische Universität Dortmund}

\begin{document}
\begin{frame}
	\titlepage
\end{frame}

\begin{frame}{Inhalt}
	\tableofcontents
\end{frame}

\section{Motivation}

\begin{frame}{Motivation: Neuronale Netze zur Bildsegmentierung}
\begin{columns}
	\begin{column}{.5\textwidth}
		\begin{itemize}
			\item Neuronale Netzwerke werden oft zur Bildsegmentierung genutzt
			\item Voraussetzung: Manuell erstellte Ground Truth um das Netzwerk zu trainieren
			\end{itemize}
	\end{column}
	\begin{column}{.5\textwidth}
		\begin{figure}[H]
			\includegraphics[width=\textwidth,keepaspectratio]{koeln00.png}
			\caption{Beispiel: CityScapes Dataset\cite{Cordts_2016_CVPR}}
		\end{figure}
	\end{column}
\end{columns}
\end{frame}

\begin{frame}{Motivation: (Fehlende) Ground Truths}
Ground Truth nicht immer vorhanden: Beispiel Marsoberfläche
\begin{itemize}
	\item Zu großer Datensatz
	\item Notwendigkeit von Experten
	\item[$\Rightarrow$] Manuelle Erstellung nicht kostengünstig oder zeiteffizient möglich
\end{itemize}
\medskip
Lösungsansatz:
\begin{itemize}
	\item Anfangs zufällige Klassifizierung durch Segmentierungsalgorithmus weiter optimieren
\end{itemize}
\end{frame}

\section{Verwandte Arbeiten}

\begin{frame}{Verwandte Arbeiten: Segmentierung nach} Kanezaki\cite{kanezaki2018_unsupervised_segmentation}]
Asako Kanezaki; Unsupervised Image Segmentation by Backpropagation\cite{kanezaki2018_unsupervised_segmentation}:
\begin{columns}
	\begin{column}{.5\textwidth}
		\begin{itemize}
			\item Unüberwachtes Lernen der Segmentierung
			\item Anfangs zufällige Ergebnisse werden mit Clusteringalgorithmus vereint
			\item Zielfunktion: Softmax-Loss zwischen Ergebnis des NN und des optimierten Ergebnisses
			\item NN wird auf diese Zielfunktion hin optimiert (Backpropagation)
		\end{itemize}
	\end{column}
	\begin{column}{.5\textwidth}
		\begin{figure}
			\includegraphics[width=\textwidth,keepaspectratio]{kanezaki.png}
			\caption{Vorgehensweise nach Kanezaki\cite{kanezaki2018_unsupervised_segmentation}}
		\end{figure}
	\end{column}
\end{columns}
\end{frame}

\begin{frame}{Verwandte Arbeiten: \textit{Crater Detection via CNNs}\cite{2016arXiv160100978C}}
	
\end{frame}

\section{Vorgehensweise}

\section{Referenzen}

\begin{frame}[shrink=30]{Referenzen}
	\bibliographystyle{gerabbrv3}
	\bibliography{presentation}
\end{frame}

	
\end{document}