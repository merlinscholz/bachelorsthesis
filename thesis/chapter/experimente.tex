\chapter{Experimente}
\label{chap:experimente}

\section{Datens"atze}
\label{sec:datensätze}

\section{Evaluierung}
\label{sec:evaluierung}

Zur Evaluierung und zum Vergleich der einzelnen Ansätze werden Metriken benötigt, die die verschiedenen Stärken und Schwächen der jeweiligen Lösungen zu einem Wert zusammenfassen. Hierbei ist zu beachten, dass ein Großteil der in anderen Arbeiten verwendeten Metriken wie \bspw der $F_1$-Score an dieser Stelle nicht geeignet sind, da diese voraussetzen, dass das jeweilige Bild nicht nur segmentiert, sondern diese Segmente auch benannt sind. Da der hier vorgestellte Ansatz unbenannte Cluster erzeugt, weichen wir auf den Rand Index\cite{randindex} aus. Dieser ist wie folgt definiert:
% TODO Herleiten
\[R=\frac{a+b}{\binom{n}{2}}\]

Der F1-Score wird lediglich dazu genutzt, um eine berechnete Segmentierung, in der manuell die entsprechenden Label, die Krater darstellen, mit anderen Methoden zur Kratererkennung zu vergleichen. (\vgl Unterabschnitt~\ref{ssec:f1_crater})

\section{Resultate}
\label{sec:resultate}

\section{Vergleich}
\label{sec:vergleich}

\subsection{F1-Score zur Kratererkennung}
\label{ssec:f1_crater}

In \tablename~\ref{tab:comparision} werden die in Abschnitt~\ref{sec:craterdetection} vorgestellten Methoden zur Kratererkennung verglichen.
Die F1-Scores für Urbach '09 und Bandeira '10 wurden aus den in \cite{bandeira_10} angegebenen \textit{True Positive}, \textit{False Positive} und \textit{False Negative} berechnet. Für Bandeira '12 konnte kein F1-Score berechnet werden, da dort die benötigten Werte nicht veröffentlicht wurden.

\begin{table}[h!]
	\centering
	\begin{tabular}{l | s s s}
		Region & Urbach '09 & Bandeira '10 & Cohen '16\\
		\hline
		Westen & 67,95\% & 85,33\% & 88,78\% \\
		Mitte  & 69,63\% & 79,35\% & 88,81\% \\
		Osten  & 78,30\% & 86,09\% & 90,29\% \\
	\end{tabular}
	\caption{Die F1-Scores der vorgestellten Methoden}
	\label{tab:comparision}
\end{table}