\chapter{Experimente}
\label{chap:experimente}

\section{Datens"atze}
\label{sec:datensätze}

\section{Evaluierung}
\label{sec:evaluierung}

Zur Evaluierung und zum Vergleich der einzelnen Ansätze werden Metriken benötigt, die die verschiedenen Stärken und Schwächen der jeweiligen Lösungen zu einem Wert zusammenfassen. Hierbei ist zu beachten, dass ein Großteil der in anderen Arbeiten verwendeten Metriken wie \bspw der $F_1$-Score an dieser Stelle nicht geeignet sind, da diese voraussetzen, dass das jeweilige Bild nicht nur segmentiert, sondern diese Segmente auch benannt sind. Da der hier vorgestellte Ansatz unbenannte Cluster erzeugt, weichen wir auf den Rand Index\cite{randindex} aus. Dieser ist wie folgt definiert:
% TODO Herleiten
\[R=\frac{a+b}{\binom{n}{2}}\]


\section{Resultate}
\label{sec:resultate}

\section{Vergleich}
\label{sec:vergleich}