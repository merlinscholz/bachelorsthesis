\chapter{Einleitung}
\label{chap:einleitung}

\section{Motivation}

Neuronale Netzwerke werden in den letzten Jahren häufig zur Objekterkennung und Bildsegmentierung genutzt. Ein weit bekanntes Anwendungsbeispiel findet sich in der Entwicklung des autonomen Fahrens, bei welcher das Bild in verschiedene Klassen wie \zB Fahrbahn, Ampel oder Passanten segmentiert wird. Dies geschieht alles unter der Voraussetzung einer Ground Truth, aus der das Netzwerk lernen kann, wie es die jeweiligen Klassen erkennen kann.

Diese Ground Truths werden meist von Hand erschaffen. So können diese im Straßenverkehr von jedermann erstellt werden, während in spezifischeren Anwendungsgebieten meist Domänenexperten dazu benötigt werden. Dies ist auch der Fall bei der Analyse der Marsoberfläche. Aufgrund der schlichten Größe des Datensatzes, und der Notwendigkeit von Experten in diesem Fachgebiet, ist es hier weder leicht, noch zeiteffizient oder kostengünstig möglich, manuell eine Ground Truth zu entwickeln.

\section{Rahmen der Arbeit}