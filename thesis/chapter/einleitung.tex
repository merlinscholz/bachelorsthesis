\chapter{Einleitung}
\label{chap:einleitung}

\section{Motivation}
\label{sec:motivation}

Neuronale Netzwerke werden in den letzten Jahren häufig zur Objekterkennung und Bildsegmentierung genutzt. Ein weit bekanntes Anwendungsbeispiel findet sich in der Entwicklung des autonomen Fahrens, bei welcher das Bild in verschiedene Klassen wie \zB Fahrbahn, Ampel oder Passant segmentiert wird. Dies geschieht für gewöhnlich durch überwachtes Lernen, also unter der Voraussetzung einer Ground Truth, aus denen das Netzwerk lernen kann, wie es die jeweiligen Klassen erkennen kann.

Diese Ground Truths werden meist von Hand erschaffen. So können diese im Straßenverkehr von jedermann erstellt werden, während in spezifischeren Anwendungsgebieten meist Domänenexperten dazu benötigt werden. Dies ist auch der Fall bei der Analyse der Marsoberfläche. Aufgrund der schlichten Größe des Datensatzes, und der Notwendigkeit von Experten in diesem Fachgebiet, ist es hier weder leicht, noch zeiteffizient oder kostengünstig möglich, manuell eine Ground Truth zu entwickeln.

Es besteht zwar die Möglichkeit bereits vorhandene Ground Truths zu nutzen: So könnten \bspw Aufnahmen der (stärker erkundeten) Mondoberfläche oder gar Erdaufnahmen angepasst werden, um als Trainingsdatensatz zur Analyse der Marsoberfläche zu dienen. Dies wirft allerdings neue Probleme auf. Bei \bspw sich stark verändernden Lichtverhältnissen, Aufnahmewinkeln, \oa kann ein vortrainiertes Neuronales Netz meistens keine optimalen Ergebnisse erzielen, eben weil es nicht auf diese neuen Rahmenbedingungen trainiert wurde.

Neben dem überwachten Lernen existiert auch das unüberwachte Lernen. Dieses arbeitet ohne Ground Truth, stattdessen muss dem Algorithmus eine andere Methode übermittelt werden, anhand welcher er erkennen kann, wie korrekt ein Ergebnis ist. Dies geschieht \zB über einen Algorithmus, der das Problem zwar ohne neuronale Netze löst, allerdings keine so hohe Genauigkeit besitzt.

\section{Rahmen der Arbeit}
\label{sec:rahmen}

In dieser Arbeit wird versucht, das Problem der Segmentierung der Marsoberfläche durch unüberwachtes Lernen zu lösen. Begonnen wird mit der Implementierung eines einfachen Algorithmus zur Segmentierung auf Basis des Ansatzes von nach \cite{kanezaki_18}, welcher klassische Clusteringalgorithmen zur Initialisierung nutzt, die anschließend durch ein neuronales Netzwerk weiter verfeinert werden (\vgl Unterabschnitt~\ref{ssec:kanezaki}). Dieser Prozess wird über verschiedene Ansätze, wie \zB das Ersetzen der Initialisierungsmethode, die Veränderung der Netzwerkarchitektur, oder die Analyse des Einflusses bestimmter Hyperparameter weiter optimiert. Anschließend werden diese Ergebnisse miteinander verglichen, und die beste Kombination von ihnen ausgewählt, um mit alternativen Algorithmen zur Bildsegmentierung --~im Allgemeinen oder domänenspezifisch~-- verglichen zu werden.