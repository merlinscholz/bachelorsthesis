\chapter{Grundlagen}
\label{chap:grundlagen}

\section{Analyse der Marsoberfl"ache}
\label{sec:analysedermarsoberflache}


\section{Contex Camera des Mars Reconnaisance Orbiters}
\label{sec:mroctx}

\section{Bildsegmentierung}
\label{sec:segmentierung}

Um ein Eingabebild, wie \zB eine Aufnahme der Marsoberfläche, in verschiedene Bereiche aufteilen zu können, bedarf es eines Algorithmus zur Bildsegmentierung. Wie in \cite{bildsegmentierung} aufgeführt, beschreibt der Überbegriff der Bildsegmentierung oftmals den Ablauf des Pre-Processing eines Eingabebildes, gefolgt von der eigentlichen \textquote{Segmentierung der Bilder in Regionen, Objekte mit geschlossenen Konturen oder Liniensegmente} und, je nach Anwendungsfall, der anschließenden Einordnung der einzelnen Segmente.

In dem hier beschriebenen Anwendungsbereich besteht das Pre-Processing aus dem Herunterladen der Eingabedaten von der Website des \textit{Planetary Data System} der NASA, insbesondere deren \textit{Imaging Node} (\hyperlink{https://pds-imaging.jpl.nasa.gov/}{https://pds-imaging.jpl.nasa.gov/}). Die dort gehosteten Aufnahmen der MRO Context Camera sind unverarbeitet und befinden sich in einem speziell für diesen Zweck erstellten Dateiformat, % TODO Zitieren
sodass diese erst konvertiert, und anschließend so verarbeitet werden müssen, dass normale Bilder des Mars entstehen.
Diese Verarbeitung besteht aus dem Laden der Metadaten der einzelnen Aufnahmen, gefolgt von deren Kalibrierung, der Entfernung der Even/Odd Detector Stripes, und der Konvertierung in das GeoTIFF format. Dieses kann von dem eigentlichen Analyse-Algorithmus eingelesen und weiter verarbeitet werden.

\section{Neuronale Netze}
\label{sec:neuronalenetze}