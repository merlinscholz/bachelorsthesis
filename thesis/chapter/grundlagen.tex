\chapter{Grundlagen}
\label{chap:grundlagen}

\section{Analyse der Marsoberfl"ache}
\label{sec:analysedermarsoberflache}


\section{Contex Camera des Mars Reconnaisance Orbiters}
\label{sec:mroctx}

\section{Bildsegmentierung}
\label{sec:segmentierung}

Um ein Eingabebild, wie \zB eine Aufnahme der Marsoberfläche, in verschiedene Bereiche aufteilen zu können, bedarf es eines Algorithmus zur Bildsegmentierung. Wie in \cite{bildsegmentierung} aufgeführt, beschreibt der Überbegriff der Bildsegmentierung oftmals den Ablauf des Pre-Processing eines Eingabebildes, gefolgt von der eigentlichen \textquote{Segmentierung der Bilder in Regionen, Objekte mit geschlossenen Konturen oder Liniensegmente} und, je nach Anwendungsfall, der anschließenden Einordnung der einzelnen Segmente.

\section{Pre-Processing}
\label{sec:preprocessing}

In dem hier beschriebenen Anwendungsbereich besteht das Pre-Processing aus dem Herunterladen der Eingabedaten von der Website des \textit{Planetary Data System} der NASA, insbesondere deren \textit{Imaging Node}\cite{pds}. Die dort gehosteten Aufnahmen der MRO Context Camera sind größtenteils unverarbeitet und befinden sich in einem speziell für diesen Zweck erstellten Dateiformat\footnote{\url{https://pds-imaging.jpl.nasa.gov/data/mro/mars\_reconnaissance\_orbiter/ctx/mrox\_0001/document/ctxsis.pdf}}\footnote{\url{https://pds-imaging.jpl.nasa.gov/data/mro/mars\_reconnaissance\_orbiter/ctx/mrox\_0001/document/archsis.pdf}}, sodass diese erst konvertiert und anschließend so verarbeitet werden müssen, dass für diesen Zweck besser geeignete Bilddateien entstehen.
Diese Verarbeitung besteht aus dem Herunterladen der Metadaten der einzelnen Aufnahmen, gefolgt von deren Kalibrierung, der Entfernung der Even/Odd Detector Stripes, und der Konvertierung in das GeoTIFF format. Dieses kann von dem eigentlichen Analyse-Algorithmus eingelesen und weiter verarbeitet werden.

Zur eigentlichen Verarbeitung wird das ISIS Image Processing Software Package\cite{isis} in Kombination mit GNU Parallel\cite{gnuparallel} genutzt.

\section{Neuronale Netze}
\label{sec:neuronalenetze}

Obwohl einfache Versionen neuronaler Netze schon seit vielen Jahrzehnten in der Informatik genutzt werden, wurden in den letzten Jahren durch Deep Neural Networks neue Durchbrüche in verschiedensten Bereichen erzielt. Dieser Umschwung ist darauf zurückzuführen, dass mit dem Fortschritt der Technik nun auch große Netzwerke effizient ausgeführt werden können.